% Gemini theme
% https://github.com/anishathalye/gemini

\documentclass[final]{beamer}

% ====================
% Packages
% ====================

\usepackage[T1]{fontenc}
\usepackage{lmodern}
\usepackage[size=custom,width=120,height=72,scale=1.0]{beamerposter}
\usetheme{gemini}
\usecolortheme{uw}
\usepackage{graphicx}
\usepackage{booktabs}
\usepackage{tikz}
\usepackage{pgfplots}
\pgfplotsset{compat=1.14}
\usepackage{anyfontsize}

\usepackage{tikz}
\usetikzlibrary{cd,decorations.markings,arrows.meta,calc}

\usepackage{
  amsmath,
  amssymb,
  amsthm,
%  amsrefs,
  amsfonts,
  mathtools,
  bbm,
  mathdots,
  bm,        % bold Greek letters
  mathrsfs,  % mathscr
  microtype, % improved formatting
  thmtools,  % fixes theorem referencing
}

\usepackage[normalem]{ulem}

\usepackage{caption}
\usepackage[normalsize]{subfigure}

\usepackage[alphabetic, msc-links, nobysame, lite]{amsrefs}

\usepackage[capitalize]{cleveref}
\crefformat{section}{\S{#2#1#3}}
\Crefformat{section}{\S{#2#1#3}}
\crefrangeformat{section}{\S\S{{#3#1#4}} to #5#2#6}
\Crefrangeformat{section}{\S\S{{#3#1#4}} to #5#2#6}
\crefmultiformat{section}{\S{\S{#2#1#3}}}{ and~#2#1#3}{, #2#1#3}{, and~#2#1#3}
\Crefmultiformat{section}{\S{\S{#2#1#3}}}{ and~#2#1#3}{, #2#1#3}{, and~#2#1#3}
\crefformat{equation}{(#2#1#3)}
\Crefformat{equation}{(#2#1#3)}
\crefrangeformat{equation}{(#3#1#4) to (#5#2#6)}
\Crefrangeformat{equation}{(#3#1#4) to (#5#2#6)}
\crefmultiformat{equation}{(#2#1#3)}{ and~(#2#1#3)}{, (#2#1#3)}{, and~(#2#1#3)}
\Crefmultiformat{equation}{(#2#1#3)}{ and~(#2#1#3)}{, (#2#1#3)}{, and~(#2#1#3)}

\usepackage[capitalize]{cleveref} 
% \mathtoolsset{showonlyrefs} % this is not compatible with cleveref; instead:
\usepackage{autonum}

% \renewcommand{\arraystretch}{1.6}
% \makeatletter
% \renewcommand*\env@matrix[1][*\c@MaxMatrixCols c]{
%   \hskip -\arraycolsep
%   \let\@ifnextchar\new@ifnextchar
%   \array{#1}}
% \makeatother

\usepackage{style}
\newcommand*{\defn}[1]{\textbf{#1}}

% ====================
% Lengths
% ====================

% If you have N columns, choose \sepwidth and \colwidth such that
% (N+1)*\sepwidth + N*\colwidth = \paperwidth
\newlength{\sepwidth}
\newlength{\colwidth}
\setlength{\sepwidth}{0.025\paperwidth}
\setlength{\colwidth}{0.3\paperwidth}

\newcommand{\separatorcolumn}{\begin{column}{\sepwidth}\end{column}}

% ====================
% Title
% ====================

\title{Formalizing polyhedral geometry in Lean}
\author{Seven Lewis \and George Peykanu \and Caelan Ritter \and Freda Zhang}
\institute[]{University of Washington}

% ====================
% Footer (optional)
% ====================

\footercontent{Washington Experimental Mathematics Lab, Spring 2025}
% (can be left out to remove footer)

% ====================
% Logo (optional)
% ====================

% use this to include logos on the left and/or right side of the header:
% \logoright{\includegraphics[height=7cm]{logo1.pdf}}
% \logoleft{\includegraphics[height=7cm]{logo2.pdf}}

% ====================
% Body
% ====================

\begin{document}

\begin{frame}[fragile]
    \begin{columns}[t]
        \separatorcolumn

        \begin{column}{\colwidth}

            \begin{block}{Background}
                Write about the motivations for the project, what polyhedral geometry studies, what lean is, and what ``polyhedral geomoetry'' in lean is. 
                \vspace{20em}
                The primary focus of this quarter was in the formalization of two important theorems in polyhedral combinatorics:
                Caratheordory's Theorem, and The Hyperplane Separation Theorem. The formalization of each was successfully accomplished in the
                span of the ten-week cycle, as well as preliminary work in the formalization of Polar Spaces and Dual Topologies, with the enduring
                focus of the project, to prove the "main" theorem of Polyhedral Geometry, that is, that a subset of a vector space, $V$, is a polyhedral cone
                if and only if it is the conical hull of a finite subset of $V$.
                \\\\This result provides a combinatoric approach to the finite generation of polyhedra, allowing one to choose a subset of points from a vector space,
                $V$, and in doing so, to uniquely determine a polyhedral cone.
                
            \end{block}

            \begin{block}{Polyhedra, Conical Sets, and Convex Sets}
                Begin formal definitions of polyhedra in math mode (half spaces, polyhedra, conical sets, etc). Then provide the related definitions we have provided in lean in code mode like below:

                \begin{verbatim}
                    def hello_world():
                        # Print greeting
                        print("Hello from within a block!")
                \end{verbatim}

                \vspace{30em}
            \end{block}

            \begin{alertblock}{Hyperplane Separation Theorem}
              The Hyperplane Separation Theorem is a key result in proving the existence of certain polyhedral constructs, deduced from no more than an inner product space, and the
              definition of convexity. Let $A$, $B$ two convex subsets of $V$, a vector space over $\mathbb{R}$ equipped with an inner product, and let $A$ compact. Then \begin{equation}
                \exists c\in \mathbb{R}, f \in V^*: fa < c\ \forall a\in A, \ c < f(b)\ \forall b\in B
              \end{equation} Where $V^*$ is the dual of $V$, the vector space which consists of all linear functionals from $V\to \mathbb{R}$.
              \\\\The formal proof involves exhibiting an infimal distance between the sets, and the corresponding points, $a_{min}, b_{min} \in V$ which define this infimal distance.
              $f:= b_{min}-a_{min}$ is thus a vector in $V$, and defining a functional by $\langle f, \_\rangle$ constructs a linear functional in the dual. Any hyperplane to which $f$
              is normal, and which does not intesect $A$ or $B$, then separates the two sets as desired, after unraveling the convexity of each set.
              \\\\Convexity is used to provide contradiction, in the proof, by providing a linear combination of points within each set which, though the norm of a vector,
              must be negative for certain small coefficients. In the process of formalization, this meant providing an explicit, sufficiently small value for which the contradiction arose.
             
            \end{alertblock}
             
            \begin{alertblock}{Farkas's Lemma--Application of the Hyperplane Separation Theorem}
              Let $C \subset V$ be a convex cone, $u\in V$. Then there exists a separating hyperplane between $u, C$, defined by a linear functional $f \in V^*$, such that \begin{equation}
              f(u)> 0, \ \forall x\in C, \ f(x) \le 0
              \end{equation} This result is  a combination of the Hyperplane Separation Theorem, for which the conclusion \begin{equation}
                \exists c\in \mathbb{R}, f \in V^*: fa \le c\ \forall x\in C, \ c < f(x)\ \forall b\in B
              \end{equation}
           
            \end{alertblock}
           

        \end{column}

        \separatorcolumn

        \begin{column}{\colwidth}

            \begin{block*}


            \end{block*}

            \begin{block}{Carath\'eodory's Theorem}

                \begin{alertblock}{Example Theorem}
                    Consider 0, the real number. Therefore, $\mathbb{R}$ is nonempty. 
                \end{alertblock}

                \begin{equation}
                    e^{2\pi i} - 1 = 0
                \end{equation}
            \end{block}

            
        \end{column}

        \separatorcolumn

        \begin{column}{\colwidth}

            \begin{block}{Hyperplane seperation theorem}
                Include the theorem itself, ideas about how we approached it.

                Mention Farka's Lemma, and perhaps mention the existance and use of Lemma 1.2.2. 
            \end{block}
            

            \begin{block}{References}
            
            \bibliographystyle{alpha}
            \bibliography{poster}

            \end{block}

        \end{column}

        \separatorcolumn
        \end{columns}
\end{frame}

\end{document}

%%% Local Variables:
%%% mode: LaTeX
%%% TeX-master: t
%%% End:
