% Gemini theme
% https://github.com/anishathalye/gemini

\documentclass[final]{beamer}

% ====================
% Packages
% ====================

\usepackage[T1]{fontenc}
\usepackage{lmodern}
\usepackage[size=custom,width=120,height=72,scale=1.0]{beamerposter}
\usetheme{gemini}
\usecolortheme{uw}
\usepackage{graphicx}
\usepackage{booktabs}
\usepackage{tikz}
\usepackage{pgfplots}
\pgfplotsset{compat=1.14}
\usepackage{anyfontsize}

\setmonofont{DejaVu Sans Mono}[Scale=MatchLowercase]

\usepackage{tikz}
\usetikzlibrary{cd,decorations.markings,arrows.meta,calc}

\usepackage{
  amsmath,
  amssymb,
  amsthm,
  amsfonts,
  mathtools,
  bbm,
  mathdots,
  bm,        % bold Greek letters
  mathrsfs,  % mathscr
  microtype, % improved formatting
  thmtools,  % fixes theorem referencing
}

\usepackage[normalem]{ulem}

\usepackage{caption}
\usepackage[normalsize]{subfigure}

\usepackage[msc-links, nobysame, lite, url]{amsrefs}

\usepackage[capitalize]{cleveref}
\crefformat{section}{\S{#2#1#3}}
\Crefformat{section}{\S{#2#1#3}}
\crefrangeformat{section}{\S\S{{#3#1#4}} to #5#2#6}
\Crefrangeformat{section}{\S\S{{#3#1#4}} to #5#2#6}
\crefmultiformat{section}{\S{\S{#2#1#3}}}{ and~#2#1#3}{, #2#1#3}{, and~#2#1#3}
\Crefmultiformat{section}{\S{\S{#2#1#3}}}{ and~#2#1#3}{, #2#1#3}{, and~#2#1#3}
\crefformat{equation}{(#2#1#3)}
\Crefformat{equation}{(#2#1#3)}
\crefrangeformat{equation}{(#3#1#4) to (#5#2#6)}
\Crefrangeformat{equation}{(#3#1#4) to (#5#2#6)}
\crefmultiformat{equation}{(#2#1#3)}{ and~(#2#1#3)}{, (#2#1#3)}{, and~(#2#1#3)}
\Crefmultiformat{equation}{(#2#1#3)}{ and~(#2#1#3)}{, (#2#1#3)}{, and~(#2#1#3)}

\usepackage[capitalize]{cleveref} 
% \mathtoolsset{showonlyrefs} % this is not compatible with cleveref; instead:
\usepackage{autonum}

% \renewcommand{\arraystretch}{1.6}
% \makeatletter
% \renewcommand*\env@matrix[1][*\c@MaxMatrixCols c]{
%   \hskip -\arraycolsep
%   \let\@ifnextchar\new@ifnextchar
%   \array{#1}}
% \makeatother

\usepackage{style}
\newcommand*{\defn}[1]{\textbf{#1}}

% ====================
% Lengths
% ====================

% If you have N columns, choose \sepwidth and \colwidth such that
% (N+1)*\sepwidth + N*\colwidth = \paperwidth
\newlength{\sepwidth}
\newlength{\colwidth}
\setlength{\sepwidth}{0.025\paperwidth}
\setlength{\colwidth}{0.3\paperwidth}

\newcommand{\separatorcolumn}{\begin{column}{\sepwidth}\end{column}}

% ====================
% Title
% ====================

\title{Formalizing polyhedral geometry in Lean}
\author{Seven Lewis \and George Peykanu \and Caelan Ritter \and Freda Zhang}
\institute[]{University of Washington}

% ====================
% Footer (optional)
% ====================

\footercontent{Washington Experimental Mathematics Lab, Spring 2025}
% (can be left out to remove footer)

% ====================
% Logo (optional)
% ====================

% use this to include logos on the left and/or right side of the header:
% \logoright{\includegraphics[height=7cm]{logo1.pdf}}
% \logoleft{\includegraphics[height=7cm]{logo2.pdf}}

% ====================
% Body
% ====================

\begin{document}

\begin{frame}[fragile]
    \begin{columns}[t]
        \separatorcolumn

        \begin{column}{\colwidth}

            \begin{block}{Background}
           
                The primary focus of this quarter was in the formalization of two important theorems in polyhedral combinatorics:
                Caratheordory's Theorem, and The Hyperplane Separation Theorem. The formalization of each was successfully accomplished in the
                span of the ten-week cycle, as well as preliminary work in the formalization of Polar Spaces and Dual Topologies, with the enduring
                focus of the project, to prove the "main" theorem of Polyhedral Geometry, that is, that a subset of a vector space, $V$, is a polyhedral cone
                if and only if it is the conical hull of a finite subset of $V$.
                \\\\This result provides a combinatoric approach to the finite generation of polyhedra, allowing one to choose a subset of points from a vector space,
                $V$, and in doing so, to uniquely determine a polyhedral cone.
                
            \end{block}

            \begin{block}{Finite Polyhedra, Conical Sets, and Convex Sets}
                Have $V = \RR^d$ be the ambient finite-dimensional vector space. The \textbf{halfspace} is commonly defined as the set of points on one side of an affine hyperplane:
                $$ \{\mathbf{x} \in V \;|\; \mathbf{ax} \leq b\}$$
                where $\mathbf{a} \in \RR^d$ is the normal vector of the hyperplane with $b \in \RR$ as the affine constant term.

                We define the \textbf{finite polyhedron} as the intersection of finitely many halfspaces $\{H_1,H_2,\dots,H_n\}$:
                $$ \bigcap_{1 \leq i \leq n} H_i. $$
                A \textbf{convex} set $S \subseteq V$ is any set with the property that given two elements $\mathbf{x}, \mathbf{y} \in S$, we have
                $$ (1 - t)\mathbf{x} + t\mathbf{y} \in S \quad \text{ for all real } t \in [0,1].$$
                A \textbf{convex hull} is the closure of any $S \subseteq V$ under the property of convexity. Equivalently, $\text{conv}S$ is the smallest convex set of $V$ which contains $S$. 

                A \textbf{polytope} can be defined a bounded polyhedron or, equivalently, the convex hull of finitely many points in $\RR^d$. We provide a proof of this in our Lean formalization. 
                
                It turns out that \textbf{conical sets} are a nice generalization of the idea of a \textbf{convex set}. As a result, much of our formulation relies on the definition of the \textbf{conical combination} and \textbf{conical set} which we wil define below:
                
                NOT FINISHED
                               
            \end{block}
           
            
        \end{column}

        \separatorcolumn

        \begin{column}{\colwidth}

            \begin{block*}


            \end{block*}

            \begin{block}{Carath\'eodory's Theorem}

                \begin{alertblock}{Example Theorem}
                    Consider 0, the real number. Therefore, $\mathbb{R}$ is nonempty. 
                \end{alertblock}

                \begin{equation}
                    e^{2\pi i} - 1 = 0
                \end{equation}


                \begin{tikzpicture}[scale=2]
                    % 1) The six true vertices of a regular hexagon of radius 1
                    \foreach \i in {0,...,5} {
                        \coordinate (P\i) at ({2*cos(60*\i)},{2*sin(60*\i)});
                    }

                    % 2) Draw each extended edge in dotted style and name it E<i>
                    \foreach \i/\j in {0/1,1/2,2/3,3/4,4/5,5/0} {
                        \draw[name path=E\i,dotted]
                        ($(P\i)!-2!(P\j)$) -- ($(P\j)!-2!(P\i)$);
                    }

                    % 3) Over‐draw the actual hexagon edges in solid
                    \draw[thick] (P0)--(P1)--(P2)--(P3)--(P4)--(P5)--cycle;

                    % (Optional) put dots at the true hexagon vertices, too
                    \foreach \i in {0,...,5} {
                        \fill (P\i) circle (1pt);
                    }

                    % 4) Shade the triangle P0–P2–P4
                    \fill[purple!30,opacity=0.5] (P0) -- (P1) -- (P4) -- cycle;

                    % 5) Construct two medians of that triangle and name their intersection G
                    \coordinate (m02) at ($ (P0)!0.5!(P2) $);   % midpoint of P0–P2
                    \coordinate (m24) at ($ (P2)!0.5!(P4) $);   % midpoint of P2–P4

                    \path[name path=med1] (P4) -- (m02);  
                    \path[name path=med2] (P0) -- (m24);

                    \path[name intersections={of=med1 and med2, by=G}];

                    % 6) Draw a dot at the centroid G
                    \coordinate (Gshift) at,(,(G)+(-0.3cm,0.15cm),);\quad
                    \fill (Gshift) circle (1pt);
                \end{tikzpicture}
            \end{block}

            
        \end{column}

        \separatorcolumn

        \begin{column}{\colwidth}

            \begin{block}{Hyperplane seperation theorem}
                            \begin{alertblock}{Hyperplane Separation Theorem}
              The Hyperplane Separation Theorem is a key result in proving the existence of certain polyhedral constructs, deduced from no more than an inner product space, and the
              definition of convexity. Let $A$, $B$ two convex subsets of $V$, a vector space over $\mathbb{R}$ equipped with an inner product, and let $A$ compact. Then \begin{equation}
                \exists c\in \mathbb{R}, f \in V^*: fa < c\ \forall a\in A, \ c < f(b)\ \forall b\in B
              \end{equation} Where $V^*$ is the dual of $V$, the vector space which consists of all linear functionals from $V\to \mathbb{R}$.
              \\\\The formal proof involves exhibiting an infimal distance between the sets, and the corresponding points, $a_{min}, b_{min} \in V$ which define this infimal distance.
              $f:= b_{min}-a_{min}$ is thus a vector in $V$, and defining a functional by $\langle f, \_\rangle$ constructs a linear functional in the dual. Any hyperplane to which $f$
              is normal, and which does not intesect $A$ or $B$, then separates the two sets as desired, after unraveling the convexity of each set.
              \\\\Convexity is used to provide contradiction, in the proof, by providing a linear combination of points within each set which, though the norm of a vector,
              must be negative for certain small coefficients. In the process of formalization, this meant providing an explicit, sufficiently small value for which the contradiction arose.

              \begin{verbatim}
              theorem hyperplane_separation  (A B : Set V) (hA : Convex ℝ A) (hB : Convex ℝ B) (hclosed : IsClosed A ∧ IsClosed B ) (hNempty : A.Nonempty ∧ B.Nonempty) (hA_Bounded: IsBounded A) (hAB : Disjoint A B) : ∃ (f : V →ₗ[ℝ] ℝ) (c : ℝ), (∀ a ∈ A, f a < c) ∧ (  ∀ b ∈ B, c < f b)
            \end{verbatim}
             
            \end{alertblock}

            
             
            \begin{alertblock}{Farkas's Lemma--Application of the Hyperplane Separation Theorem}
              Let $C \subset V$ be a convex cone, $u\in V$. Then there exists a separating hyperplane between $u, C$, defined by a linear functional $f \in V^*$, such that \begin{equation}
              f(u)> 0, \ \forall x\in C, \ f(x) \le 0
              \end{equation} This result is  a combination of the Hyperplane Separation Theorem, for which the conclusion \begin{equation}
                \exists c\in \mathbb{R}, f \in V^*: fa \le c\ \forall x\in C, \ c < f(x)\ \forall b\in B
              \end{equation}
           
            \end{alertblock}
        \end{block}
        \begin{block}{References}
        \nocite{liunotes}
        \begin{bibdiv}
        \begin{biblist}
          \bib{liunotes}{webpage}{
              title={Polyhedral combinatorics},
              author={Liu, Gaku},
              accessdate={3 June 2023},
              url={https://sites.google.com/view/gakuliu/home}
            }
        \end{biblist}
        \end{bibdiv}
        \end{block}
\end{column}

\separatorcolumn
\end{columns}
\end{frame}
\end{document}

%%% Local Variables:
%%% mode: LaTeX
%%% TeX-master: t
%%% End:
