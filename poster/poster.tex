% Gemini theme
% https://github.com/anishathalye/gemini

\documentclass[final]{beamer}

% ====================
% Packages
% ====================

\usepackage[T1]{fontenc}
\usepackage{lmodern}
\usepackage[size=custom,width=120,height=72,scale=1.0]{beamerposter}
\usetheme{gemini}
\usecolortheme{uw}
\usepackage{graphicx}
\usepackage{booktabs}
\usepackage{tikz}
\usepackage{pgfplots}
\pgfplotsset{compat=1.14}
\usepackage{anyfontsize}

\usepackage{tikz}
\usetikzlibrary{cd,decorations.markings,arrows.meta,calc}

\usepackage{
  amsmath,
  amssymb,
  amsthm,
%  amsrefs,
  amsfonts,
  mathtools,
  bbm,
  mathdots,
  bm,        % bold Greek letters
  mathrsfs,  % mathscr
  microtype, % improved formatting
  thmtools,  % fixes theorem referencing
}

\usepackage[normalem]{ulem}

\usepackage{caption}
\usepackage[normalsize]{subfigure}

\usepackage[alphabetic, msc-links, nobysame, lite]{amsrefs}

\usepackage[capitalize]{cleveref}
\crefformat{section}{\S{#2#1#3}}
\Crefformat{section}{\S{#2#1#3}}
\crefrangeformat{section}{\S\S{{#3#1#4}} to #5#2#6}
\Crefrangeformat{section}{\S\S{{#3#1#4}} to #5#2#6}
\crefmultiformat{section}{\S{\S{#2#1#3}}}{ and~#2#1#3}{, #2#1#3}{, and~#2#1#3}
\Crefmultiformat{section}{\S{\S{#2#1#3}}}{ and~#2#1#3}{, #2#1#3}{, and~#2#1#3}
\crefformat{equation}{(#2#1#3)}
\Crefformat{equation}{(#2#1#3)}
\crefrangeformat{equation}{(#3#1#4) to (#5#2#6)}
\Crefrangeformat{equation}{(#3#1#4) to (#5#2#6)}
\crefmultiformat{equation}{(#2#1#3)}{ and~(#2#1#3)}{, (#2#1#3)}{, and~(#2#1#3)}
\Crefmultiformat{equation}{(#2#1#3)}{ and~(#2#1#3)}{, (#2#1#3)}{, and~(#2#1#3)}

\usepackage[capitalize]{cleveref} 
% \mathtoolsset{showonlyrefs} % this is not compatible with cleveref; instead:
\usepackage{autonum}

% \renewcommand{\arraystretch}{1.6}
% \makeatletter
% \renewcommand*\env@matrix[1][*\c@MaxMatrixCols c]{
%   \hskip -\arraycolsep
%   \let\@ifnextchar\new@ifnextchar
%   \array{#1}}
% \makeatother

\usepackage{style}
\newcommand*{\defn}[1]{\textbf{#1}}

% ====================
% Lengths
% ====================

% If you have N columns, choose \sepwidth and \colwidth such that
% (N+1)*\sepwidth + N*\colwidth = \paperwidth
\newlength{\sepwidth}
\newlength{\colwidth}
\setlength{\sepwidth}{0.025\paperwidth}
\setlength{\colwidth}{0.3\paperwidth}

\newcommand{\separatorcolumn}{\begin{column}{\sepwidth}\end{column}}

% ====================
% Title
% ====================

\title{Formalizing polyhedral geometry in Lean}
\author{Seven Lewis \and George Peykanu \and Caelan Ritter \and Freda Zhang}
\institute[]{University of Washington}

% ====================
% Footer (optional)
% ====================

\footercontent{Washington Experimental Mathematics Lab, Spring 2025}
% (can be left out to remove footer)

% ====================
% Logo (optional)
% ====================

% use this to include logos on the left and/or right side of the header:
% \logoright{\includegraphics[height=7cm]{logo1.pdf}}
% \logoleft{\includegraphics[height=7cm]{logo2.pdf}}

% ====================
% Body
% ====================

\begin{document}

\begin{frame}[fragile]
    \begin{columns}[t]
        \separatorcolumn

        \begin{column}{\colwidth}

            \begin{block}{Background}
                Write about the motivations for the project, what polyhedral geometry studies, what lean is, and what ``polyhedral geomoetry'' in lean is. 
                \vspace{20em}
                
            \end{block}

            \begin{block}{Polyhedra, Conical Sets, and Convex Sets}
                Begin formal definitions of polyhedra in math mode (half spaces, polyhedra, conical sets, etc). Then provide the related definitions we have provided in lean in code mode like below:

                \begin{verbatim}
                    def hello_world():
                        # Print greeting
                        print("Hello from within a block!")
                \end{verbatim}

                \vspace{30em}
            \end{block}
        \end{column}

        \separatorcolumn

        \begin{column}{\colwidth}

            \begin{block*}


            \end{block*}

            \begin{block}{Carath\'eodory's Theorem}


            \end{block}

            
        \end{column}

        \separatorcolumn

        \begin{column}{\colwidth}

            \begin{block}{Hyperplane seperation theorem}
                Include the theorem itself, ideas about how we approached it.

                Mention Farka's Lemma, and perhaps mention the existance and use of Lemma 1.2.2. 
            \end{block}
            

            \begin{block}{References}
            
            \bibliographystyle{alpha}
            \bibliography{poster}

            \end{block}

        \end{column}

        \separatorcolumn
        \end{columns}
\end{frame}

\end{document}

%%% Local Variables:
%%% mode: LaTeX
%%% TeX-master: t
%%% End:
